%\documentclass[]{article}
\documentclass[10pt,a4paper]{article}

%% Added By Lucktroy
\usepackage{booktabs}
\usepackage{mdwlist}
\renewenvironment{description}{
  \begin{basedescript}{\desclabelstyle{\pushlabel}\desclabelwidth{8em}}
}{
  \end{basedescript}
}
%%

\usepackage{geometry} 		% 設定邊界
\geometry{
  top=1in,
  inner=1in,
  outer=1in,
  bottom=1in,
  headheight=3ex,
  headsep=2ex
}
\usepackage{tgpagella} % Use the TeX Gyre Pagella font throughout the document
\usepackage[T1]{fontenc}
\usepackage{lmodern}
\usepackage{amssymb,amsmath}
\usepackage{ifxetex,ifluatex}
\usepackage{fixltx2e} % provides \textsubscript
% use upquote if available, for straight quotes in verbatim environments
\IfFileExists{upquote.sty}{\usepackage{upquote}}{}
\ifnum 0\ifxetex 1\fi\ifluatex 1\fi=0 % if pdftex
  \usepackage[utf8]{inputenc}
\else % if luatex or xelatex
  \usepackage{fontspec} 	% 允許設定字體
  \usepackage{xeCJK} 		% 分開設置中英文字型
  \setCJKmainfont{Adobe Kaiti Std} 	% 設定中文字型
  %\setmainfont{Georgia} 	% 設定英文字型
  %\setromanfont{Georgia} 	% 字型
  %\setmonofont{Courier New}
  %\linespread{1.2}\selectfont 	% 行距
  %\XeTeXlinebreaklocale "zh" 	% 針對中文自動換行
  %\XeTeXlinebreakskip = 0pt plus 1pt % 字與字之間加入0pt至1pt的間距,確保左右對整齊
  %\parindent 0em 		% 段落縮進
  %\setlength{\parskip}{20pt} 	% 段落之間的距離
  \ifxetex
    \usepackage{xltxtra,xunicode}
  \fi
  \defaultfontfeatures{Mapping=tex-text,Scale=MatchLowercase}
  \newcommand{\euro}{€}
\fi
% use microtype if available
\IfFileExists{microtype.sty}{\usepackage{microtype}}{}
\ifxetex
  \usepackage[setpagesize=false, % page size defined by xetex
              unicode=false, % unicode breaks when used with xetex
              xetex]{hyperref}
\else
  \usepackage[unicode=true]{hyperref}
\fi
\hypersetup{breaklinks=true,
            bookmarks=true,
            pdfauthor={},
            pdftitle={},
            colorlinks=true,
            urlcolor=blue,
            linkcolor=magenta,
            pdfborder={0 0 0}}
\urlstyle{same}  % don't use monospace font for urls
%\setlength{\parindent}{0pt}
%\setlength{\parskip}{6pt plus 2pt minus 1pt}
\setlength{\emergencystretch}{3em}  % prevent overfull lines

\title{\huge 在OSX平台上的XeLaTeX中文測試} % 設置標題,使用巨大字體
\author{FoolEgg.com} 		% 設置作者
\date{February 2013} 		% 設置日期
\usepackage{titling}
\setlength{\droptitle}{-8em} 	% 將標題移動至頁面的上面

\usepackage{fancyhdr}
\usepackage{lastpage}
\pagestyle{fancyplain}

\setcounter{secnumdepth}{0}

\author{}
\date{}

\begin{document}

\section{\texttt{Junbo Zhang}}\label{section}

\begin{table}[h]
\begin{tabular*}{\textwidth}{l @{\extracolsep{\fill}} r}
Department of Computer Science and Engineering & (852) 5519-5158 \\
%Hong Kong Science Park & \href{mailto:jbzhang@cse.cuhk.edu.hk}{jbzhang@cse.cuhk.edu.hk} \\
The Chinese University of Hong Kong & \href{mailto:jbzhang@cuhk.edu.hk}{jbzhang@cuhk.edu.hk} \\
Shatin, Hong Kong & \href{mailto:zjb2046@gmail.com}{zjb2046@gmail.com} %%\url{http://www.lucktroy.org}
\end{tabular*}
\end{table}


\iffalse
School of Information Science and Technology :
\href{mailto:jbzhang@my.swjtu.edu.cn}{jbzhang@my.swjtu.edu.cn}

\begin{description}
\itemsep1pt\parskip0pt\parsep0pt
\item[Southwest Jiaotong University, Chengdu, China]
\url{http://www.lucktroy.org}
\end{description}

\subsection{Short Bio}\label{short-bio}

Junbo Zhang is currently a Research Assistant at The Chinese University
of Hong Kong and a Research Intern at Huawei Noah's Ark Lab (Hong Kong).
He received his B.Eng. degree in Telecommunication Engineering (the Mao
Yisheng Honors Class) from Southwest Jiaotong University. He was a
visiting Ph.D.~student at the Department of Computer Science, Georgia
State University (Feb. 2012 - Feb. 2013) and an Intern at the Belgian
Nuclear Research Centre (SCK-CEN), Belgium (Aug. 2011 - Sep. 2011). He
is a student member of both ACM and China Computer Federation. \fi

\subsection{Education}\label{education}

\begin{description}
\item[09/2009 - present]
\textbf{Southwest Jiaotong University}, Chengdu, Sichuan, China

Ph.D.~candidate in Compute Science, advised by
\href{http://userweb.swjtu.edu.cn/Userweb/trli30/index.htm}{Prof.~Tianrui
Li} \& \href{http://www.cs.gsu.edu/pan/}{Prof.~Yi Pan}

\emph{Special Grade Scholarship for 5 years}
\item[09/2005 - 06/2009]
\textbf{Southwest Jiaotong University}, Chengdu, Sichuan, China

Bachelor of Telecommunication Engineering in
\href{http://my.qy1896.com/en/}{\textbf{the Mao Yisheng Honors Class}}

\emph{Undergraduate Scholarship for 4 years}
\end{description}

\iffalse
09/2005 - present : PhD Candidate in School of Information Science and
Technology : \textbf{Southwest Jiaotong University}, Chengdu, China :
Supervisors:
\href{http://userweb.swjtu.edu.cn/Userweb/trli30/index.htm}{Prof.~Tianrui
Li} \& \href{http://www.cs.gsu.edu/pan/}{Prof.~Yi Pan}

\begin{description}
\itemsep1pt\parskip0pt\parsep0pt
\item[09/2005 - 06/2009]
Bachelor of Telecommunication Engineering

Southwest Jiaotong University, Chengdu, China
\end{description}

\fi

\subsection{Experience}\label{experience}

\iffalse
5/2013 - present : Research Assistant at The Chinese University of Hong
Kong, Hong Kong : \& Research Intern at
\href{http://www.noahlab.com.hk/}{Huawei Noah's Ark Lab (Hong Kong)},
Hong Kong \fi

\begin{description}
\item[05/2013 - present]
\href{http://www.noahlab.com.hk/}{\textbf{Huawei Noah's Ark Lab}},
Shatin, New Territories, Hong Kong

Research Intern with \href{http://www.weifan.info/}{Dr.~Wei Fan}

\emph{Research and develop large-scale deep learning and feature
engineering algorithms.}

\begin{itemize}
\itemsep1pt\parskip0pt\parsep0pt
\item
  \small Proposed a novel deep learning model named SUGAR. SUGAR
  regularizes the network construction by utilizing similarity or
  dissimilarity constraints between data pairs, rather than
  sample-specific annotations. Such side information is more flexible
  and greatly mitigates the workload of annotators. Unlike prior works,
  SUGAR decouples the supervision information and intrinsic data
  structure. It includes two heterogeneous networks, capturing
  supervision and unsupervised data structure, respectively.
  \texttt{The work has been presented at}
  \href{http://www.kdd.org/kdd2014/}{\emph{KDD 2014}}. \texttt{Code:}
  \url{https://github.com/lucktroy/sugar}.
\end{itemize}

\begin{itemize}
\itemsep1pt\parskip0pt\parsep0pt
\item
  \small Proposed a novel algorithm (FSP-MbT) for mining discriminative
  frequent sequential patterns via model-based search tree. It is the
  core component of a subsequent prediction model, which is used for the
  consumer complaint problem in the telecommunications industry. The
  related prediction system is being deployed and tested for the
  real-world application, serving millions of consumers.
  \texttt{Patent Pending}.
\end{itemize}

\begin{itemize}
\itemsep1pt\parskip0pt\parsep0pt
\item
  \small Proposed a locally linear deep learning model (LLDL) for
  large-scale stellar spectrum recognition. The characteristics of
  stellar spectra include: the massive volume, ultrahigh dimensionality
  (\textgreater{}5K), and significantly low signal/noise ratio
  (\textless{}10). Using GPU, stochastic gradient descent (SGD), and
  Dropout, LLDL can model the locally linear structures found in the
  stellar spectra. Experimental results show that LLDL achieves up to
  100\% improvements over both popular deep neural networks and shallow
  SVM and Logistic Regression. \texttt{Paper is under review}.
\end{itemize}
\end{description}

\iffalse
: - \small Research on Transformational Learning (TL, the generation of
deep learning). Using deep learning techniques with decision tree,
model-based search tree or graphical models, TL can learn structure \&
high-level knowledge representations from most kinds of data, including
sequences, graphs, images, etc. \texttt{The draft is being prepared}.
\fi

\begin{description}
\itemsep1pt\parskip0pt\parsep0pt
\item[05/2013 - present]
\textbf{The Chinese University of Hong Kong}, Shatin, New Territories,
Hong Kong

Research Assistant with \href{http://www.cs.cuhk.hk/~cslui/}{Prof.~John
C.S. Lui}

\emph{Research and develop large-scale deep learning and feature
engineering algorithms.}
\end{description}

\iffalse
5/2013 - present : Research Assistant at The Chinese University of Hong
Kong, Hong Kong \fi

\begin{description}
\item[02/2012 - 02/2013]
\textbf{Georgia State University}, Atlanta, GA, USA

Research Assistant with \href{http://www.cs.gsu.edu/pan/}{Prof.~Yi Pan}

\emph{Research and develop parallel algorithms, large-scale algorithms
in cloud computing and GPU cluster.}

\begin{itemize}
\itemsep1pt\parskip0pt\parsep0pt
\item
  \small Participated as a mentor in 8-week NSF REU Undergraduate Summer
  Research Program.
\end{itemize}

\begin{itemize}
\itemsep1pt\parskip0pt\parsep0pt
\item
  \small Proposed large-scale feature selection algorithms, provide
  HADOOP and SPARK implementations.
\end{itemize}

\begin{itemize}
\itemsep1pt\parskip0pt\parsep0pt
\item
  \small Proposed a composite relation for data fusion \& a novel
  parallel matrix algorithm using multi-GPU.
\end{itemize}
\item[08/2011 - 09/2011]
\textbf{Belgian Nuclear Research Centre (SCK-CEN)}, Mol, Belgium

Intern with
\href{http://be.linkedin.com/pub/klaas-van-der-meer/b/a4b/393}{Dr.~Klaas
van der Meer}

\emph{Research and develop incremenal and parallel algorithm for feature
selection and knowledge acquisition.}
\end{description}

\subsection{Awards \& Honors}\label{awards-honors}

\renewenvironment{description}{
  \begin{basedescript}{\desclabelstyle{\pushlabel}\desclabelwidth{5em}}
}{
  \end{basedescript}
}

\small

\begin{description}
\item[2012, 2013]
National Scholarship, China.
\item[2012]
``\href{http://baike.baidu.com/view/644025.htm}{Si Shi Yang Hua}
竢实扬华'' Medal in \href{http://www.swjtu.edu.cn/}{Southwest Jiaotong
University}, 2012. (Top 1/1000, the students' top honor of Southwest
Jiaotong University).
\item[2009-2014]
Special Grade Scholarship for Ph.D.~Students, Southwest Jiaotong
University.
\item[2012]
First Prize in the $9^{\text{th}}$ National Postgraduate Mathematical
Contest in Modeling, China.
\item[2009, 2011]
Second Prize in the $6^{\text{th}}$ \& $8^{\text{th}}$ National
Postgraduate Mathematical contest in Modeling, China.
\item[2010]
Second Prize (Top 10 in person) in the $2^{\text{nd}}$ ``Huawei Cup''
Innovation Programming Contest, China.
\item[2007, 2008]
Second Prize (Top 10, Team leader) in the $1^{\text{st}}$ \&
$2^{\text{nd}}$ Sichuan Provincial Programming Contest, China.
\item[2007]
Second Prize (the $12^{\text{th}}$ Place in person) in the TopCoder
Sichuan Provincial Contest, China.
\item[2006, 2007]
The President Scholarship, School of Information Science and Technology,
Southwest Jiaotong University, China.
\item[2005-2009]
Undergraduate Scholarship, Southwest Jiaotong University.
\end{description}

\iffalse

\subsection{Research Interests}\label{research-interests}

\begin{itemize}
\itemsep1pt\parskip0pt\parsep0pt
\item
  Deep Learning, Representation Learning, Feature Engineering
\item
  Cloud Computing, Distributed Computing, High Performance Computing
\item
  Big Data Mining
\item
  Granular Computing
\end{itemize}

\fi

\subsection{Publications}\label{publications}

\small

\begin{enumerate}
\def\labelenumi{\arabic{enumi}.}
\item
  \textbf{Junbo Zhang}, Guangjian Tian, Yadong Mu, Wei Fan.\\
  \href{http://dx.doi.org/10.1145/2623330.2623618}{\emph{Supervised Deep
  Learning with Auxiliary Networks.}}\\ Proceedings of the 20th ACM
  SIGKDD Conference on Knowledge Discovery and Data Mining
  (\href{http://www.kdd.org/kdd2014/}{\textbf{KDD 2014}}), New York,
  USA, 2014, pp.~353-361.\\ (AR: 151/1036 = 14.6\%)
\item
  \textbf{Junbo Zhang}, Jian-Syuan Wong, Yi Pan, Tianrui Li.\\
  \href{http://dx.doi.org/10.1109/TKDE.2014.2330821}{\emph{A Parallel
  Matrix-based Method for Computing Approximations in Incomplete
  Information Systems.}}\\ IEEE Transactions on Knowledge and Data
  Engineering
  (\href{http://www.computer.org/portal/web/tkde}{\textbf{TKDE}}),
  vol.~27, no. 2, pp.~326-339, 2015.
\item
  \textbf{Junbo Zhang}, Tianrui Li, Yi Pan, Chuan Luo, Fei Teng.\\
  \href{http://www.jos.org.cn/ch/reader/view_abstract.aspx?file_no=4590}{\emph{A
  Parallel and Incremental Algorithm for Updating Knowledge Based on
  Rough Sets in Cloud Computing Platform.}}\\ Accepted for publication
  in \href{http://www.jos.org.cn/ch/index.aspx}{\textbf{Journal of
  Software}}. (in Chinese)
\item
  \textbf{Junbo Zhang}, Jian-Syuan Wong, Tianrui Li, Yi Pan.\\
  \href{http://dx.doi.org/10.1016/j.ijar.2013.08.003}{\emph{A Comparison
  of Parallel Large-scale Knowledge Acquisition Using Rough Set Theory
  on Different MapReduce Runtime Systems.}}\\
  \href{http://www.journals.elsevier.com/international-journal-of-approximate-reasoning}{\textbf{International
  Journal of Approximate Reasoning}}, vol.~55, no. 3, pp.~896-907, 2014.
\item
  \textbf{Junbo Zhang}, Tianrui Li, Hongmei Chen.\\
  \href{http://dx.doi.org/10.1016/j.ins.2013.08.016}{\emph{Composite
  Rough Sets for Dynamic Data Mining.}}\\
  \href{http://www.journals.elsevier.com/information-sciences}{\textbf{Information
  Sciences}}, vol.~257, pp.~81-100, 2014.
\item
  \textbf{Junbo Zhang}, Dong Xiang, Tianrui Li, Yi Pan.\\
  \href{http://ieeexplore.ieee.org/xpl/articleDetails.jsp?tp=\&arnumber=6449402\&contentType=Journals+\%26+Magazines\&queryText\%3DM2M\%3A+A+simple+Matlab-to-MapReduce+translator+for+Cloud+Computing}{\emph{M2M:
  A Simple Matlab-to-MapReduce Translator for Cloud Computing.}}\\
  \href{http://qhxb.lib.tsinghua.edu.cn/english/}{\textbf{Tsinghua
  Science and Technology}}, vol 18, no. 1, pp.~1-9, 2013.
\item
  \textbf{Junbo Zhang}, Tianrui Li, Da Ruan, Zizhe Gao, Chengbing
  Zhao.\\ \href{http://dx.doi.org/10.1016/j.ins.2011.12.036}{\emph{A
  Parallel Method for Computing Rough Set Approximations.}}\\
  \href{http://www.journals.elsevier.com/information-sciences}{\textbf{Information
  Sciences}}, vol.~194, pp.~209-223, 2012.
\item
  \textbf{Junbo Zhang}, Tianrui Li, Da Ruan, Dun Liu.\\
  \href{http://dx.doi.org/10.1016/j.ijar.2012.01.001}{\emph{Rough Sets
  Based Matrix Approaches with Dynamic Attribute Variation in Set-valued
  Information Systems.}}\\
  \href{http://www.journals.elsevier.com/international-journal-of-approximate-reasoning}{\textbf{International
  Journal of Approximate Reasoning}}, vol.~53, no. 4, pp.~620-635, 2012.
\item
  \textbf{Junbo Zhang}, Tianrui Li, Da Ruan, Dun Liu.\\
  \href{http://dx.doi.org/10.1002/int.21523}{\emph{Neighborhood Rough
  Sets for Dynamic Data Mining.}}\\
  \href{http://onlinelibrary.wiley.com/journal/10.1002/{[}ISSN{]}1098-111X}{\textbf{International
  Journal of Intelligent Systems}}, vol.~27, no. 4, pp.~317-342, 2012.
\item
  Dun Liu, Tianrui Li, \textbf{Junbo Zhang}.\\
  \href{http://dx.doi.org/10.1016/j.ijar.2014.05.009}{\emph{A Rough
  Set-based Incremental Approach for Learning Knowledge in Dynamic
  Incomplete Information Systems.}}\\
  \href{http://www.journals.elsevier.com/international-journal-of-approximate-reasoning}{\textbf{International
  Journal of Approximate Reasoning}}, vol 55, no. 8, pp.~1764--1786,
  2014.
\item
  Yi Pan, \textbf{Junbo Zhang}.\\
  \href{http://www.ftrai.org/xe/index.php?mid=joc_published\&category=37964\&search_keyword=section\&search_target=title\&document_srl=38459}{\emph{Parallel
  Programming on Cloud Computing Platforms: Challenges and
  Solutions.}}\\ \href{http://www.ftrai.org/joc/}{\textbf{KITCS/FTRA
  Journal of Convergence}}, vol.~3, no. 4, pp.~23-28,2012.
\item
  Dun Liu, Tianrui Li, Da Ruan, \textbf{Junbo Zhang}.\\
  \href{http://dx.doi.org/10.1007/s10898-010-9607-8}{\emph{Incremental
  learning optimization on knowledge discovery in dynamic business
  intelligent systems.}}\\
  \href{http://www.springer.com/business+\%26+management/operations+research/journal/10898}{\textbf{Journal
  of Global Optimization}}, vol.~51, no. 27: pp.~325-344, 2011.
\end{enumerate}

\subsection{Programming and System
Skills}\label{programming-and-system-skills}

\begin{description}
\item[Operating System]
Linux-based OS (e.g., Cent OS, Ubuntu), Mac OS, Windows XP, Windows 7/8
\item[Programming]
C/C++, \href{http://www.nvidia.com/object/cuda_home_new.html}{CUDA C}
(for NVIDIA GPUs), Python, Java, Shell and various languages with
practical experiences.
\item[Others]
\href{http://hadoop.apache.org/}{Hadoop}/\href{http://spark.apache.org/}{Spark}/\href{http://mapreduce.stanford.edu/}{Phoenix++}/\href{http://www.iterativemapreduce.org/}{Twister}
(MapReduce-like Softwares), \href{https://github.com/GraphChi}{GraphChi}
(Graph Computing System),
\href{http://deeplearning.net/software/theano/}{Theano}/\href{http://deeplearning.net/software/pylearn2/}{Pylearn2}
(GPU-based Deep Learning in Python),
\href{http://scikit-learn.org/stable/}{Scikit-learn} (Machine Learning
in Python), \href{http://www.cs.waikato.ac.nz/ml/weka/}{Weka} (Data
Mining Software in Java), Matlab, \LaTeX, MS Office (e.g., Visio, Word,
Excel, PowerPoint)
\end{description}

\subsection{Projects}\label{projects}

\begin{description}
\item[11/2012 - present]
Research on dynamic knowledge discovery techniques and efficient
algorithms under granular computing, the Fostering Foundation for the
Excellent Ph.D.~Dissertation of Southwest Jiaotong University, China.
Leader and principal investigator.
\item[10/2012 - 10/2013]
Dynamic knowledge discovery system based on rough sets in cloud
computing environments, the Science and Technology Planning Project of
Sichuan Province, China. Leader and principal investigator.
\item[10/2011 - 09/2012]
Research on dynamic knowledge discovery techniques under granular
computing and probabilistic rough sets, and its fast algorithms based on
cloud computing, the Doctoral Innovation Funding Project of Southwest
Jiaotong University, China. Leader and principal investigator.
\item[10/2011 - 09/2012]
Research on dynamic knowledge discovery system based on cloud computing
and rough sets, the Young Software Innovation Foundation of Sichuan
Province, China. Leader and principal investigator.
\item[05/2012 - 07/2012]
8-week NSF REU Undergraduate Summer Research Program hosted by the
Department of Computer Science, Georgia State University, the National
Science Foundation, USA. Participated as a mentor.
\item[01/2012 - present]
Research on dynamic updating knowledge theories and algorithms based on
granular computing, the National Natural Science Foundation of China.
\item[01/2012 - present]
Research on composite rough set models and algorithms of knowledge
discovery, the National Natural Science Foundation of China.
\item[01/2009 - 12/2011]
Research on incremental learning theories and methods based on granular
computing, the National Natural Science Foundation of China.
\end{description}

\subsection{Professional Activities}\label{professional-activities}

\subsubsection{Review for Journal}\label{review-for-journal}

\begin{itemize}
\itemsep1pt\parskip0pt\parsep0pt
\item
  \href{http://tkdd.acm.org/}{ACM Transactions on Knowledge Discovery
  from Data (TKDD)}
\item
  \href{http://www.computer.org/portal/web/tc}{IEEE Transactions on
  Computers (TOC)}
\item
  \href{http://www.computer.org/portal/web/tcbb}{IEEE/ACM Transactions
  on Computational Biology and Bioinformatics (TCBB)}
\item
  \href{http://ieeexplore.ieee.org/xpl/RecentIssue.jsp?punumber=7728}{IEEE
  Transactions on NanoBioscience}
\item
  \href{http://www.editorialmanager.com/wwwj/}{World Wide Web Journal
  (WWWJ)}
\item
  \href{http://www.inderscience.com/jhome.php?jcode=ijcc}{International
  Journal of Cloud Computing (IJCC)}
\item
  \href{http://www.inderscience.com/jhome.php?jcode=ijbra}{International
  Journal of Bioinformatics Research and Applications (IJBRA)}
\item
  \href{http://www.atlantis-press.com/publications/ijcis/}{International
  Journal of Computational Intelligence Systems (IJCIS)}
\end{itemize}

\subsubsection{Extracurricular
Activities}\label{extracurricular-activities}

\renewenvironment{description}{
  \begin{basedescript}{\desclabelstyle{\pushlabel}\desclabelwidth{4em}}
}{
  \end{basedescript}
}

\begin{description}
\item[04/2012]
IBM Cloud Academy CON, NC, Research Triangle Park (RTP), North Carolina,
USA.
\item[12/2010]
Workshop on Frontiers of Data Management, Soochow University, Suzhou,
China.
\item[11/2010]
Workshop on Massive Data Mining and Knowledge Discovery, Southwest
Jiaotong University, Chengdu, China.
\item[11/2010]
China Computer Federation Advanced Disciplines Lectures (the 11th issue)
- Massive Data Mining and Knowledge Discovery, Southwest Jiaotong
University, Chengdu, China.
\item[08/2010]
Intel Software College - Multi-core Programming for Academia, Fudan
University, Shanghai, China.
\item[07/2010]
National Graduate Summer School - Data Intensive Computing and
Unstructured Data Management, Renmin University of China, Beijing,
China.
\item[12/2009]
Workshop on Massive Data Mining and Knowledge Discovery, Southwest
Jiaotong University, Chengdu, China.
\item[08/2009]
National Graduate Summer School (Dragon Star Plan) - Data Mining,
Zhejiang University, Hangzhou, China.
\end{description}

\subsection{Languages}\label{languages}

Chinese (native), English (fluent)

\end{document}

